%%%%%%%%%%%%%%%%%%%%%%%%%%%%%%%%%%%%%%%%%%%%%%%%%%%%%%%%%%%%%%%%%%%%%%
%                                                       
%%%%%%%%%%%%%%%%%%%%%%%%%%%%%%%%%%%%%%%%%%%%%%%%%%%%%%%%%%%%%%%%%%%%%%
\documentclass[a4paper,11pt]{article}
\usepackage{amsfonts,amssymb}
\usepackage{fullpage}
\usepackage[francais]{babel}
\usepackage[T1]{fontenc}
\usepackage[latin1]{inputenc}
\usepackage{graphicx}

% \textheight = 25cm
% \voffset = -0.5cm
\parindent=0cm

%%%%%%%%%%%%%%%%%%%%%%%%%%%%%%%%%%%%%%%%%%%%%
%      _      _           _   
%   __| | ___| |__  _   _| |_ 
%  / _` |/ _ \ '_ \| | | | __|
% | (_| |  __/ |_) | |_| | |_ 
%  \__,_|\___|_.__/ \__,_|\__|
%                            
%%%%%%%%%%%%%%%%%%%%%%%%%%%%%%%%%%%%%%%%%%%%%
\begin{document}
\selectlanguage{francais}

\vspace{-3cm}
\begin{center}
  \rule{15cm}{0.1mm} \\

  \vspace{0.5cm}
  {\huge \textbf{\textsc{Algorithmique Combinatoire}}} \\
  Master 1 Informatique  \\  
  {\Large TP 3 --  programmation dynamique} \\
  \rule{15cm}{0.1mm}\\

\end{center}

Ci-joint sur la courbe de la figure~\ref{fig:perfChaine} les temps de
calcul ou les temps estim�s du calcul du meilleur temps de travers�e
d'une usine automobile, faite de deux cha�nes de montage. Ce calcul se
fait soit par un algorithme r�curcif, soir par un algorithme issu du
paradigme de la programmation dynamique. Pour ce faire j'ai eu recourt
� la l'utilisation de la graine ??? afin de garantir de
reproductibilit�  des applications stochastiques. 

\begin{figure}[h]
  \centerline{%
    \includegraphics[angle=-90,width=12cm]{perfFactory.pdf}}
  \caption{Temps de calcul en fonction du mode de calcul (r�cursif ou it�ratif) et de la taille des cha�nes de montage.}
  \label{fig:perfChaine}
\end{figure}


\end{document}



%%% Local Variables: 
%%% mode: latex
%%% mode: flyspell
%%% ispell-local-dictionary: "francais" 
%%% TeX-master: "partiel2004"
% LocalWords: 
%%% End: 
